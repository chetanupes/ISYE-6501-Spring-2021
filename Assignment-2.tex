% Options for packages loaded elsewhere
\PassOptionsToPackage{unicode}{hyperref}
\PassOptionsToPackage{hyphens}{url}
%
\documentclass[
]{article}
\usepackage{lmodern}
\usepackage{amssymb,amsmath}
\usepackage{ifxetex,ifluatex}
\ifnum 0\ifxetex 1\fi\ifluatex 1\fi=0 % if pdftex
  \usepackage[T1]{fontenc}
  \usepackage[utf8]{inputenc}
  \usepackage{textcomp} % provide euro and other symbols
\else % if luatex or xetex
  \usepackage{unicode-math}
  \defaultfontfeatures{Scale=MatchLowercase}
  \defaultfontfeatures[\rmfamily]{Ligatures=TeX,Scale=1}
\fi
% Use upquote if available, for straight quotes in verbatim environments
\IfFileExists{upquote.sty}{\usepackage{upquote}}{}
\IfFileExists{microtype.sty}{% use microtype if available
  \usepackage[]{microtype}
  \UseMicrotypeSet[protrusion]{basicmath} % disable protrusion for tt fonts
}{}
\makeatletter
\@ifundefined{KOMAClassName}{% if non-KOMA class
  \IfFileExists{parskip.sty}{%
    \usepackage{parskip}
  }{% else
    \setlength{\parindent}{0pt}
    \setlength{\parskip}{6pt plus 2pt minus 1pt}}
}{% if KOMA class
  \KOMAoptions{parskip=half}}
\makeatother
\usepackage{xcolor}
\IfFileExists{xurl.sty}{\usepackage{xurl}}{} % add URL line breaks if available
\IfFileExists{bookmark.sty}{\usepackage{bookmark}}{\usepackage{hyperref}}
\hypersetup{
  pdftitle={Assignment 2},
  hidelinks,
  pdfcreator={LaTeX via pandoc}}
\urlstyle{same} % disable monospaced font for URLs
\usepackage[margin=1in]{geometry}
\usepackage{color}
\usepackage{fancyvrb}
\newcommand{\VerbBar}{|}
\newcommand{\VERB}{\Verb[commandchars=\\\{\}]}
\DefineVerbatimEnvironment{Highlighting}{Verbatim}{commandchars=\\\{\}}
% Add ',fontsize=\small' for more characters per line
\usepackage{framed}
\definecolor{shadecolor}{RGB}{248,248,248}
\newenvironment{Shaded}{\begin{snugshade}}{\end{snugshade}}
\newcommand{\AlertTok}[1]{\textcolor[rgb]{0.94,0.16,0.16}{#1}}
\newcommand{\AnnotationTok}[1]{\textcolor[rgb]{0.56,0.35,0.01}{\textbf{\textit{#1}}}}
\newcommand{\AttributeTok}[1]{\textcolor[rgb]{0.77,0.63,0.00}{#1}}
\newcommand{\BaseNTok}[1]{\textcolor[rgb]{0.00,0.00,0.81}{#1}}
\newcommand{\BuiltInTok}[1]{#1}
\newcommand{\CharTok}[1]{\textcolor[rgb]{0.31,0.60,0.02}{#1}}
\newcommand{\CommentTok}[1]{\textcolor[rgb]{0.56,0.35,0.01}{\textit{#1}}}
\newcommand{\CommentVarTok}[1]{\textcolor[rgb]{0.56,0.35,0.01}{\textbf{\textit{#1}}}}
\newcommand{\ConstantTok}[1]{\textcolor[rgb]{0.00,0.00,0.00}{#1}}
\newcommand{\ControlFlowTok}[1]{\textcolor[rgb]{0.13,0.29,0.53}{\textbf{#1}}}
\newcommand{\DataTypeTok}[1]{\textcolor[rgb]{0.13,0.29,0.53}{#1}}
\newcommand{\DecValTok}[1]{\textcolor[rgb]{0.00,0.00,0.81}{#1}}
\newcommand{\DocumentationTok}[1]{\textcolor[rgb]{0.56,0.35,0.01}{\textbf{\textit{#1}}}}
\newcommand{\ErrorTok}[1]{\textcolor[rgb]{0.64,0.00,0.00}{\textbf{#1}}}
\newcommand{\ExtensionTok}[1]{#1}
\newcommand{\FloatTok}[1]{\textcolor[rgb]{0.00,0.00,0.81}{#1}}
\newcommand{\FunctionTok}[1]{\textcolor[rgb]{0.00,0.00,0.00}{#1}}
\newcommand{\ImportTok}[1]{#1}
\newcommand{\InformationTok}[1]{\textcolor[rgb]{0.56,0.35,0.01}{\textbf{\textit{#1}}}}
\newcommand{\KeywordTok}[1]{\textcolor[rgb]{0.13,0.29,0.53}{\textbf{#1}}}
\newcommand{\NormalTok}[1]{#1}
\newcommand{\OperatorTok}[1]{\textcolor[rgb]{0.81,0.36,0.00}{\textbf{#1}}}
\newcommand{\OtherTok}[1]{\textcolor[rgb]{0.56,0.35,0.01}{#1}}
\newcommand{\PreprocessorTok}[1]{\textcolor[rgb]{0.56,0.35,0.01}{\textit{#1}}}
\newcommand{\RegionMarkerTok}[1]{#1}
\newcommand{\SpecialCharTok}[1]{\textcolor[rgb]{0.00,0.00,0.00}{#1}}
\newcommand{\SpecialStringTok}[1]{\textcolor[rgb]{0.31,0.60,0.02}{#1}}
\newcommand{\StringTok}[1]{\textcolor[rgb]{0.31,0.60,0.02}{#1}}
\newcommand{\VariableTok}[1]{\textcolor[rgb]{0.00,0.00,0.00}{#1}}
\newcommand{\VerbatimStringTok}[1]{\textcolor[rgb]{0.31,0.60,0.02}{#1}}
\newcommand{\WarningTok}[1]{\textcolor[rgb]{0.56,0.35,0.01}{\textbf{\textit{#1}}}}
\usepackage{graphicx,grffile}
\makeatletter
\def\maxwidth{\ifdim\Gin@nat@width>\linewidth\linewidth\else\Gin@nat@width\fi}
\def\maxheight{\ifdim\Gin@nat@height>\textheight\textheight\else\Gin@nat@height\fi}
\makeatother
% Scale images if necessary, so that they will not overflow the page
% margins by default, and it is still possible to overwrite the defaults
% using explicit options in \includegraphics[width, height, ...]{}
\setkeys{Gin}{width=\maxwidth,height=\maxheight,keepaspectratio}
% Set default figure placement to htbp
\makeatletter
\def\fps@figure{htbp}
\makeatother
\setlength{\emergencystretch}{3em} % prevent overfull lines
\providecommand{\tightlist}{%
  \setlength{\itemsep}{0pt}\setlength{\parskip}{0pt}}
\setcounter{secnumdepth}{-\maxdimen} % remove section numbering

\title{Assignment 2}
\author{}
\date{\vspace{-2.5em}}

\begin{document}
\maketitle

\hypertarget{question-3.1-a}{%
\subsection{Question 3.1 (a)}\label{question-3.1-a}}

\begin{itemize}
\tightlist
\item
  Using cross-validation (do this for the k-nearest-neighbors model; SVM
  is optional)
\end{itemize}

\hypertarget{answer-3.1-a}{%
\subsection{Answer 3.1 (a)}\label{answer-3.1-a}}

\begin{itemize}
\tightlist
\item
  \textbf{Doing cross validation manually}
\end{itemize}

\begin{Shaded}
\begin{Highlighting}[]
\CommentTok{#Importing libraries }
\KeywordTok{library}\NormalTok{(kknn)}
\KeywordTok{library}\NormalTok{(ggplot2)}

\CommentTok{#Reading the data}
\NormalTok{credit_data<-}\KeywordTok{read.table}\NormalTok{(}\StringTok{'credit_card_data-headers.txt'}\NormalTok{, }\DataTypeTok{sep =} \StringTok{""}\NormalTok{, }\DataTypeTok{header =} \OtherTok{TRUE}\NormalTok{)}

\CommentTok{#Viewing the data}
\KeywordTok{head}\NormalTok{(credit_data)}
\end{Highlighting}
\end{Shaded}

\begin{verbatim}
##   A1    A2    A3   A8 A9 A10 A11 A12 A14 A15 R1
## 1  1 30.83 0.000 1.25  1   0   1   1 202   0  1
## 2  0 58.67 4.460 3.04  1   0   6   1  43 560  1
## 3  0 24.50 0.500 1.50  1   1   0   1 280 824  1
## 4  1 27.83 1.540 3.75  1   0   5   0 100   3  1
## 5  1 20.17 5.625 1.71  1   1   0   1 120   0  1
## 6  1 32.08 4.000 2.50  1   1   0   0 360   0  1
\end{verbatim}

\begin{Shaded}
\begin{Highlighting}[]
\CommentTok{#Summarizing the data}
\KeywordTok{str}\NormalTok{(credit_data)}
\end{Highlighting}
\end{Shaded}

\begin{verbatim}
## 'data.frame':    654 obs. of  11 variables:
##  $ A1 : int  1 0 0 1 1 1 1 0 1 1 ...
##  $ A2 : num  30.8 58.7 24.5 27.8 20.2 ...
##  $ A3 : num  0 4.46 0.5 1.54 5.62 ...
##  $ A8 : num  1.25 3.04 1.5 3.75 1.71 ...
##  $ A9 : int  1 1 1 1 1 1 1 1 1 1 ...
##  $ A10: int  0 0 1 0 1 1 1 1 1 1 ...
##  $ A11: int  1 6 0 5 0 0 0 0 0 0 ...
##  $ A12: int  1 1 1 0 1 0 0 1 1 0 ...
##  $ A14: int  202 43 280 100 120 360 164 80 180 52 ...
##  $ A15: int  0 560 824 3 0 0 31285 1349 314 1442 ...
##  $ R1 : int  1 1 1 1 1 1 1 1 1 1 ...
\end{verbatim}

\begin{Shaded}
\begin{Highlighting}[]
\CommentTok{#Setting the seed for reproducibility}
\KeywordTok{set.seed}\NormalTok{(}\DecValTok{42}\NormalTok{)}

\CommentTok{#Shuffling the data randomly}
\NormalTok{credit_data<-credit_data[}\KeywordTok{sample}\NormalTok{(}\KeywordTok{nrow}\NormalTok{(credit_data)),]}

\CommentTok{#Creating 10 groups for cross validation. One can also user input to decide }
\CommentTok{#on the number of cross validation groups. Will be more automated.}
\NormalTok{group <-}\StringTok{ }\KeywordTok{cut}\NormalTok{(}\KeywordTok{seq}\NormalTok{(}\DecValTok{1}\NormalTok{,}\KeywordTok{nrow}\NormalTok{(credit_data)),}\DataTypeTok{breaks=}\DecValTok{10}\NormalTok{,}\DataTypeTok{labels=}\OtherTok{FALSE}\NormalTok{)}
\end{Highlighting}
\end{Shaded}

\begin{itemize}
\tightlist
\item
  \textbf{Building the Model}
\end{itemize}

\begin{Shaded}
\begin{Highlighting}[]
\NormalTok{pre<-}\KeywordTok{list}\NormalTok{() }\CommentTok{#Creating an empty list to store predictions}
\NormalTok{accu<-}\KeywordTok{list}\NormalTok{() }\CommentTok{#Creating an empty list to store accuracy}
\ControlFlowTok{for}\NormalTok{ (j }\ControlFlowTok{in} \DecValTok{1}\OperatorTok{:}\DecValTok{50}\NormalTok{)\{   }\CommentTok{#Loop to test different k between 1-50}
\NormalTok{  acc1=}\DecValTok{0}           \CommentTok{#Setting Accuracy to 0 for new value of k}
  \ControlFlowTok{for}\NormalTok{ (i }\ControlFlowTok{in} \DecValTok{1}\OperatorTok{:}\DecValTok{10}\NormalTok{)\{ }\CommentTok{#Loop to test k-fold cross validation}
\NormalTok{    ind<-}\KeywordTok{which}\NormalTok{(group}\OperatorTok{==}\NormalTok{i, }\DataTypeTok{arr.ind =} \OtherTok{FALSE}\NormalTok{) }\CommentTok{#This create a set using the group }
                                          \CommentTok{#defined above.}
\NormalTok{    model_knn =}\StringTok{ }\KeywordTok{kknn}\NormalTok{(R1}\OperatorTok{~}\NormalTok{.,}
                     \DataTypeTok{train=}\NormalTok{credit_data[}\OperatorTok{-}\NormalTok{ind,], }
                     \DataTypeTok{test=}\NormalTok{credit_data[ind,], }
                     \DataTypeTok{k =}\NormalTok{ j, }
                     \DataTypeTok{scale =} \OtherTok{TRUE}\NormalTok{)}
\NormalTok{    pred_knn =}\StringTok{ }\KeywordTok{fitted}\NormalTok{(model_knn)}
\NormalTok{    pre[[i]]<-}\KeywordTok{ifelse}\NormalTok{(pred_knn}\OperatorTok{>}\FloatTok{0.5}\NormalTok{,}\DecValTok{1}\NormalTok{,}\DecValTok{0}\NormalTok{)}
\NormalTok{    acc1=acc1}\OperatorTok{+}\KeywordTok{sum}\NormalTok{(pre[[i]] }\OperatorTok{==}\StringTok{ }\NormalTok{credit_data[ind,}\DecValTok{11}\NormalTok{])}\OperatorTok{/}\KeywordTok{nrow}\NormalTok{(credit_data[ind,])}
\NormalTok{  \}}
\NormalTok{  accu[[j]]=acc1}\OperatorTok{/}\DecValTok{10} \CommentTok{# Average accuracy for the jth value representing k}
\NormalTok{\}}
\end{Highlighting}
\end{Shaded}

\begin{Shaded}
\begin{Highlighting}[]
\CommentTok{#Plotting k vs accuracy}
\NormalTok{dat<-}\KeywordTok{data.frame}\NormalTok{(}\DataTypeTok{k=}\KeywordTok{list}\NormalTok{(}\DecValTok{1}\OperatorTok{:}\DecValTok{50}\NormalTok{), }\DataTypeTok{Accuracy=}\KeywordTok{unlist}\NormalTok{(accu))}
\KeywordTok{ggplot}\NormalTok{(}\DataTypeTok{data=}\NormalTok{dat,}\KeywordTok{aes}\NormalTok{(}\DataTypeTok{x=}\NormalTok{X1}\FloatTok{.50}\NormalTok{,}\DataTypeTok{y=}\NormalTok{Accuracy))}\OperatorTok{+}\KeywordTok{geom_point}\NormalTok{(}\DataTypeTok{alpha=}\FloatTok{0.5}\NormalTok{,}\DataTypeTok{color=}\StringTok{'red'}\NormalTok{)}
\end{Highlighting}
\end{Shaded}

\includegraphics{Assignment-2_files/figure-latex/unnamed-chunk-3-1.pdf}

\begin{Shaded}
\begin{Highlighting}[]
\CommentTok{#Sorting the dataframe to get best k}

\NormalTok{dat_sort<-dat[}\KeywordTok{order}\NormalTok{(}\OperatorTok{-}\NormalTok{dat}\OperatorTok{$}\NormalTok{Accuracy),]}
\NormalTok{best_k<-dat_sort[}\DecValTok{1}\NormalTok{,}\DecValTok{1}\NormalTok{]}

\KeywordTok{message}\NormalTok{(}\StringTok{'Best k: '}\NormalTok{,best_k)}
\end{Highlighting}
\end{Shaded}

\begin{verbatim}
## Best k: 5
\end{verbatim}

\begin{Shaded}
\begin{Highlighting}[]
\KeywordTok{message}\NormalTok{(}\StringTok{'Accuracy of the model using the best k: '}\NormalTok{, dat_sort[}\DecValTok{1}\NormalTok{,}\DecValTok{2}\NormalTok{])}
\end{Highlighting}
\end{Shaded}

\begin{verbatim}
## Accuracy of the model using the best k: 0.84993006993007
\end{verbatim}

\begin{itemize}
\tightlist
\item
  \textbf{Now trying the same thing using cv.kknn}
\end{itemize}

\begin{Shaded}
\begin{Highlighting}[]
\NormalTok{accu_new<-}\KeywordTok{list}\NormalTok{() }\CommentTok{#Creating an empty list to store accuracy}

\CommentTok{#Model fit}
\ControlFlowTok{for}\NormalTok{ (m }\ControlFlowTok{in} \DecValTok{1}\OperatorTok{:}\DecValTok{50}\NormalTok{)\{ }\CommentTok{#Testing different k's}
  \KeywordTok{set.seed}\NormalTok{(}\DecValTok{42}\NormalTok{)}
\NormalTok{  kNN_fit<-}\KeywordTok{cv.kknn}\NormalTok{(R1}\OperatorTok{~}\NormalTok{.,}\DataTypeTok{data=}\NormalTok{credit_data, }
                   \DataTypeTok{scale=}\OtherTok{TRUE}\NormalTok{, }\DataTypeTok{k =}\NormalTok{ m, }\DataTypeTok{kcv =} \DecValTok{10}\NormalTok{)}
\NormalTok{  pred<-}\KeywordTok{ifelse}\NormalTok{(kNN_fit[[}\DecValTok{1}\NormalTok{]][,}\DecValTok{2}\NormalTok{]}\OperatorTok{>}\FloatTok{0.5}\NormalTok{,}\DecValTok{1}\NormalTok{,}\DecValTok{0}\NormalTok{)}
\NormalTok{  accu_new[[m]]=}\KeywordTok{sum}\NormalTok{(pred}\OperatorTok{==}\NormalTok{credit_data[,}\DecValTok{11}\NormalTok{])}\OperatorTok{/}\KeywordTok{nrow}\NormalTok{(credit_data)}
\NormalTok{\}}

\CommentTok{#Plotting k vs accuracy}
\NormalTok{dat_new<-}\KeywordTok{data.frame}\NormalTok{(}\DataTypeTok{k=}\KeywordTok{list}\NormalTok{(}\DecValTok{1}\OperatorTok{:}\DecValTok{50}\NormalTok{), }\DataTypeTok{Accuracy=}\KeywordTok{unlist}\NormalTok{(accu_new))}
\KeywordTok{ggplot}\NormalTok{(}\DataTypeTok{data=}\NormalTok{dat_new,}\KeywordTok{aes}\NormalTok{(}\DataTypeTok{x=}\NormalTok{X1}\FloatTok{.50}\NormalTok{,}\DataTypeTok{y=}\NormalTok{Accuracy))}\OperatorTok{+}\KeywordTok{geom_point}\NormalTok{(}\DataTypeTok{alpha=}\FloatTok{0.5}\NormalTok{,}\DataTypeTok{color=}\StringTok{'red'}\NormalTok{)}
\end{Highlighting}
\end{Shaded}

\includegraphics{Assignment-2_files/figure-latex/unnamed-chunk-4-1.pdf}

\begin{Shaded}
\begin{Highlighting}[]
\CommentTok{#Sorting the dataframe to get best k}

\NormalTok{dat_sort_new<-dat_new[}\KeywordTok{order}\NormalTok{(}\OperatorTok{-}\NormalTok{dat_new}\OperatorTok{$}\NormalTok{Accuracy),]}
\NormalTok{best_k_new<-dat_sort_new[}\DecValTok{1}\NormalTok{,}\DecValTok{1}\NormalTok{]}

\KeywordTok{message}\NormalTok{(}\StringTok{'Best k: '}\NormalTok{,best_k_new)}
\end{Highlighting}
\end{Shaded}

\begin{verbatim}
## Best k: 7
\end{verbatim}

\begin{Shaded}
\begin{Highlighting}[]
\KeywordTok{message}\NormalTok{(}\StringTok{'Accuracy of the model using the best k: '}\NormalTok{, dat_sort_new[}\DecValTok{1}\NormalTok{,}\DecValTok{2}\NormalTok{])}
\end{Highlighting}
\end{Shaded}

\begin{verbatim}
## Accuracy of the model using the best k: 0.845565749235474
\end{verbatim}

\hypertarget{answer-3.1-b}{%
\subsection{Answer 3.1 (b)}\label{answer-3.1-b}}

\begin{itemize}
\tightlist
\item
  Splitting the data into training, validation, and test data sets (pick
  either KNN or SVM; the other is optional).
\end{itemize}

\begin{Shaded}
\begin{Highlighting}[]
\CommentTok{#We will use the caret library to do the splitting}
\KeywordTok{library}\NormalTok{(caret)}
\KeywordTok{library}\NormalTok{(kknn)}
\KeywordTok{library}\NormalTok{(ggplot2)}
\end{Highlighting}
\end{Shaded}

\begin{Shaded}
\begin{Highlighting}[]
\CommentTok{#Getting the data}
\NormalTok{credit_data<-}\KeywordTok{read.table}\NormalTok{(}\StringTok{'credit_card_data-headers.txt'}\NormalTok{, }\DataTypeTok{sep =} \StringTok{""}\NormalTok{, }\DataTypeTok{header =} \OtherTok{TRUE}\NormalTok{)}

\CommentTok{#The data will be divided with 70% Train, 15% Validation and 15% Test data}
\KeywordTok{set.seed}\NormalTok{(}\DecValTok{42}\NormalTok{)}

\CommentTok{#Using cresteDataPartion }
\NormalTok{train_index<-}\KeywordTok{createDataPartition}\NormalTok{(}\DataTypeTok{y=}\NormalTok{credit_data}\OperatorTok{$}\NormalTok{R1, }\DataTypeTok{p=}\FloatTok{0.7}\NormalTok{, }
                                \DataTypeTok{times =} \DecValTok{1}\NormalTok{, }\DataTypeTok{list =} \OtherTok{FALSE}\NormalTok{)}
\CommentTok{#Training data}
\NormalTok{train_data<-credit_data[train_index,]}

\CommentTok{#Validation and testing data, this data is further divided into two sets}
\NormalTok{rest_data<-credit_data[}\OperatorTok{-}\NormalTok{train_index,]}
\NormalTok{rest_index<-}\KeywordTok{createDataPartition}\NormalTok{(}\DataTypeTok{y=}\NormalTok{rest_data}\OperatorTok{$}\NormalTok{R1, }\DataTypeTok{p=}\FloatTok{0.5}\NormalTok{, }\DataTypeTok{times =} \DecValTok{1}\NormalTok{,}\DataTypeTok{list =} \OtherTok{FALSE}\NormalTok{)}

\CommentTok{#Validation data}
\NormalTok{val_data<-rest_data[rest_index,]}

\CommentTok{#Testing data}
\NormalTok{test_data<-rest_data[}\OperatorTok{-}\NormalTok{rest_index,]}

\CommentTok{#Using kkNN}
\NormalTok{acc<-}\KeywordTok{list}\NormalTok{() }\CommentTok{#Creating an empty list to store accuracy}

\CommentTok{#Testing the kknn method for different k values (1-50)}

\ControlFlowTok{for}\NormalTok{ (i }\ControlFlowTok{in} \DecValTok{1}\OperatorTok{:}\DecValTok{50}\NormalTok{)\{}
\NormalTok{  model_kknn<-}\KeywordTok{kknn}\NormalTok{(R1}\OperatorTok{~}\NormalTok{., }\DataTypeTok{train =}\NormalTok{ train_data, }\DataTypeTok{test =}\NormalTok{ val_data,}
                   \DataTypeTok{k=}\NormalTok{i, }\DataTypeTok{scale =} \OtherTok{TRUE}\NormalTok{)}
\NormalTok{  pred<-}\KeywordTok{fitted}\NormalTok{(model_kknn)}
\NormalTok{  pre<-}\KeywordTok{ifelse}\NormalTok{(pred}\OperatorTok{>}\FloatTok{0.5}\NormalTok{,}\DecValTok{1}\NormalTok{,}\DecValTok{0}\NormalTok{)}
\NormalTok{  acc[[i]]=}\KeywordTok{sum}\NormalTok{(pre}\OperatorTok{==}\NormalTok{val_data[,}\DecValTok{11}\NormalTok{])}\OperatorTok{/}\KeywordTok{nrow}\NormalTok{(val_data)}
\NormalTok{\}}

\CommentTok{#Choosing the best value of k from the validation accuracy. To this we put both}
\CommentTok{#accuracy and k in a data frame and plot it}

\NormalTok{score<-}\KeywordTok{data.frame}\NormalTok{(}\DataTypeTok{k=}\KeywordTok{list}\NormalTok{(}\DecValTok{1}\OperatorTok{:}\DecValTok{50}\NormalTok{), }\DataTypeTok{Accuracy=}\KeywordTok{unlist}\NormalTok{(acc))}
\KeywordTok{ggplot}\NormalTok{(}\DataTypeTok{data =}\NormalTok{ score, }\KeywordTok{aes}\NormalTok{(}\DataTypeTok{x=}\NormalTok{X1}\FloatTok{.50}\NormalTok{, }\DataTypeTok{y=}\NormalTok{Accuracy))}\OperatorTok{+}\KeywordTok{geom_point}\NormalTok{(}\DataTypeTok{alpha=}\FloatTok{0.5}\NormalTok{,}\DataTypeTok{color=}\StringTok{'red'}\NormalTok{)}
\end{Highlighting}
\end{Shaded}

\includegraphics{Assignment-2_files/figure-latex/unnamed-chunk-6-1.pdf}

\begin{Shaded}
\begin{Highlighting}[]
\CommentTok{#Finding the best model from the previous data frame}
\NormalTok{score_sort<-score[}\KeywordTok{order}\NormalTok{(}\OperatorTok{-}\NormalTok{score}\OperatorTok{$}\NormalTok{Accuracy),]}

\KeywordTok{message}\NormalTok{(}\StringTok{'Best k: '}\NormalTok{,score_sort[}\DecValTok{1}\NormalTok{,}\DecValTok{1}\NormalTok{])}
\end{Highlighting}
\end{Shaded}

\begin{verbatim}
## Best k: 9
\end{verbatim}

\begin{Shaded}
\begin{Highlighting}[]
\KeywordTok{message}\NormalTok{(}\StringTok{'Accuracy of the model using the best k: '}\NormalTok{, score_sort[}\DecValTok{1}\NormalTok{,}\DecValTok{2}\NormalTok{])}
\end{Highlighting}
\end{Shaded}

\begin{verbatim}
## Accuracy of the model using the best k: 0.887755102040816
\end{verbatim}

\begin{Shaded}
\begin{Highlighting}[]
\CommentTok{#Now testing this model  on the test data}
\NormalTok{model_test<-}\KeywordTok{kknn}\NormalTok{(R1}\OperatorTok{~}\NormalTok{.,}\DataTypeTok{train =}\NormalTok{ train_data, }\DataTypeTok{test =}\NormalTok{ test_data, }\DataTypeTok{k=}\NormalTok{score_sort[}\DecValTok{1}\NormalTok{,}\DecValTok{1}\NormalTok{],}
                 \DataTypeTok{scale =} \OtherTok{TRUE}\NormalTok{)}
\NormalTok{test_predict<-}\KeywordTok{ifelse}\NormalTok{(}\KeywordTok{fitted}\NormalTok{(model_test)}\OperatorTok{>}\FloatTok{0.5}\NormalTok{,}\DecValTok{1}\NormalTok{,}\DecValTok{0}\NormalTok{)}
\NormalTok{test_accuracy<-}\KeywordTok{sum}\NormalTok{(test_predict}\OperatorTok{==}\NormalTok{test_data[,}\DecValTok{11}\NormalTok{])}\OperatorTok{/}\KeywordTok{nrow}\NormalTok{(test_data)}

\KeywordTok{message}\NormalTok{(}\StringTok{'Accuracy of the model on the test using the best k:'}\NormalTok{, test_accuracy)}
\end{Highlighting}
\end{Shaded}

\begin{verbatim}
## Accuracy of the model on the test using the best k:0.836734693877551
\end{verbatim}

\hypertarget{question-4.1}{%
\subsection{Question 4.1}\label{question-4.1}}

Describe a situation or problem from your job, everyday life, current
events, etc., for which a clustering model would be appropriate. List
some (up to 5) predictors that you might use.

Answer:

In my job as an engineer, I investigate well performance and decide
which well may need any intervention or predict which well might go
down. We interpret the signature of the well to predict what could be
the potential problem. We use the following predictors for this
investigation.

\begin{itemize}
\tightlist
\item
  Tubing/Casing size
\item
  Reservoir Pressure
\item
  Temperature
\item
  Casing/Tubing pressure
\item
  Flow rate
\end{itemize}

I can see myself developing a clustering model where I can feed these
parameters and see what kind of problem cluster the well end up getting
into and thus proactively take necessary actions.

\hypertarget{question-4.2}{%
\subsection{Question 4.2}\label{question-4.2}}

\begin{Shaded}
\begin{Highlighting}[]
\CommentTok{#Getting the data}
\NormalTok{iris_data<-}\KeywordTok{read.table}\NormalTok{(}\StringTok{'iris.txt'}\NormalTok{, }\DataTypeTok{sep =} \StringTok{""}\NormalTok{, }\DataTypeTok{header =} \OtherTok{TRUE}\NormalTok{)}

\CommentTok{#Visualizing the data using various combination to get some intuition of the}
\CommentTok{#data distribution}
\KeywordTok{ggplot}\NormalTok{(}\DataTypeTok{data =}\NormalTok{ iris_data, }\KeywordTok{aes}\NormalTok{(Sepal.Length, }
\NormalTok{                             Sepal.Width, }
                             \DataTypeTok{color=}\NormalTok{iris_data}\OperatorTok{$}\NormalTok{Species))}\OperatorTok{+}
\StringTok{  }\KeywordTok{geom_point}\NormalTok{(}\DataTypeTok{alpha=}\DecValTok{1}\NormalTok{) }\CommentTok{#The plot shows two clusters with sertosa having its own}
\end{Highlighting}
\end{Shaded}

\includegraphics{Assignment-2_files/figure-latex/unnamed-chunk-7-1.pdf}

\begin{Shaded}
\begin{Highlighting}[]
                      \CommentTok{#clear cluster. }

\KeywordTok{ggplot}\NormalTok{(}\DataTypeTok{data =}\NormalTok{ iris_data, }\KeywordTok{aes}\NormalTok{(Petal.Length, }
\NormalTok{                             Petal.Width, }
                             \DataTypeTok{color=}\NormalTok{iris_data}\OperatorTok{$}\NormalTok{Species))}\OperatorTok{+}
\StringTok{  }\KeywordTok{geom_point}\NormalTok{(}\DataTypeTok{alpha=}\DecValTok{1}\NormalTok{) }\CommentTok{#Shows three good clusters}
\end{Highlighting}
\end{Shaded}

\includegraphics{Assignment-2_files/figure-latex/unnamed-chunk-7-2.pdf}

\begin{Shaded}
\begin{Highlighting}[]
\KeywordTok{ggplot}\NormalTok{(}\DataTypeTok{data =}\NormalTok{ iris_data, }\KeywordTok{aes}\NormalTok{(Petal.Length, }
\NormalTok{                             Sepal.Length, }
                             \DataTypeTok{color=}\NormalTok{iris_data}\OperatorTok{$}\NormalTok{Species))}\OperatorTok{+}
\StringTok{  }\KeywordTok{geom_point}\NormalTok{(}\DataTypeTok{alpha=}\DecValTok{1}\NormalTok{) }\CommentTok{#Shows three ok clusters}
\end{Highlighting}
\end{Shaded}

\includegraphics{Assignment-2_files/figure-latex/unnamed-chunk-7-3.pdf}

\begin{Shaded}
\begin{Highlighting}[]
\KeywordTok{ggplot}\NormalTok{(}\DataTypeTok{data =}\NormalTok{ iris_data, }\KeywordTok{aes}\NormalTok{(Petal.Width, }
\NormalTok{                             Sepal.Width, }
                             \DataTypeTok{color=}\NormalTok{iris_data}\OperatorTok{$}\NormalTok{Species))}\OperatorTok{+}
\StringTok{  }\KeywordTok{geom_point}\NormalTok{(}\DataTypeTok{alpha=}\DecValTok{1}\NormalTok{) }\CommentTok{#Shows three good clusters}
\end{Highlighting}
\end{Shaded}

\includegraphics{Assignment-2_files/figure-latex/unnamed-chunk-7-4.pdf}

\begin{Shaded}
\begin{Highlighting}[]
\KeywordTok{ggplot}\NormalTok{(}\DataTypeTok{data =}\NormalTok{ iris_data, }\KeywordTok{aes}\NormalTok{(Petal.Length, }
\NormalTok{                             Sepal.Width, }
                             \DataTypeTok{color=}\NormalTok{iris_data}\OperatorTok{$}\NormalTok{Species))}\OperatorTok{+}
\StringTok{  }\KeywordTok{geom_point}\NormalTok{(}\DataTypeTok{alpha=}\DecValTok{1}\NormalTok{) }\CommentTok{#Shows three clusters}
\end{Highlighting}
\end{Shaded}

\includegraphics{Assignment-2_files/figure-latex/unnamed-chunk-7-5.pdf}

\begin{Shaded}
\begin{Highlighting}[]
\KeywordTok{ggplot}\NormalTok{(}\DataTypeTok{data =}\NormalTok{ iris_data, }\KeywordTok{aes}\NormalTok{(Petal.Width, }
\NormalTok{                             Sepal.Length, }
                             \DataTypeTok{color=}\NormalTok{iris_data}\OperatorTok{$}\NormalTok{Species))}\OperatorTok{+}
\StringTok{  }\KeywordTok{geom_point}\NormalTok{(}\DataTypeTok{alpha=}\DecValTok{1}\NormalTok{) }\CommentTok{#Shows three clusters}
\end{Highlighting}
\end{Shaded}

\includegraphics{Assignment-2_files/figure-latex/unnamed-chunk-7-6.pdf}

\begin{Shaded}
\begin{Highlighting}[]
\CommentTok{#This above method of trying to plot all comibnation of predictor variables will}
\CommentTok{#not be effective when their are a lot of predictors. The conclusion from this }
\CommentTok{#is that we need somewhere between 2 to 3 clusters, but we will test for }
\CommentTok{#upto 5 clusters}

\CommentTok{#######Now evaluating kmeans with different data combinations#################}

\CommentTok{#Scaling the data}
\NormalTok{scale_iris<-}\KeywordTok{scale}\NormalTok{(iris_data[}\OperatorTok{-}\DecValTok{5}\NormalTok{])}

\CommentTok{#Tesing the algorithm with all the predictors for clusters between (2-5)}
\NormalTok{table1<-}\KeywordTok{list}\NormalTok{()}
\ControlFlowTok{for}\NormalTok{ (i }\ControlFlowTok{in} \DecValTok{2}\OperatorTok{:}\DecValTok{5}\NormalTok{)\{}
\NormalTok{  model_all<-}\KeywordTok{kmeans}\NormalTok{(}\DataTypeTok{x=}\NormalTok{scale_iris[,}\DecValTok{1}\OperatorTok{:}\DecValTok{4}\NormalTok{], }\DataTypeTok{centers =}\NormalTok{ i, }\DataTypeTok{nstart =} \DecValTok{30}\NormalTok{)}
\NormalTok{  table1[[i]]=}\KeywordTok{table}\NormalTok{(iris_data}\OperatorTok{$}\NormalTok{Species,model_all}\OperatorTok{$}\NormalTok{cluster)}
  \KeywordTok{print}\NormalTok{(table1[[i]])}
\NormalTok{\}}
\end{Highlighting}
\end{Shaded}

\begin{verbatim}
##             
##               1  2
##   setosa      0 50
##   versicolor 50  0
##   virginica  50  0
##             
##               1  2  3
##   setosa      0 50  0
##   versicolor 39  0 11
##   virginica  14  0 36
##             
##               1  2  3  4
##   setosa     25 25  0  0
##   versicolor  0  0 39 11
##   virginica   0  0 14 36
##             
##               1  2  3  4  5
##   setosa     22  0  0  0 28
##   versicolor  0  2 21 27  0
##   virginica   0 27  2 21  0
\end{verbatim}

\begin{Shaded}
\begin{Highlighting}[]
\CommentTok{#Tesing the algorithm with Petal Length & Petal Width for clusters between (2-5)}
\NormalTok{table2<-}\KeywordTok{list}\NormalTok{()}
\ControlFlowTok{for}\NormalTok{ (i }\ControlFlowTok{in} \DecValTok{2}\OperatorTok{:}\DecValTok{5}\NormalTok{)\{}
\NormalTok{  model_all<-}\KeywordTok{kmeans}\NormalTok{(}\DataTypeTok{x=}\NormalTok{scale_iris[,}\DecValTok{3}\OperatorTok{:}\DecValTok{4}\NormalTok{], }\DataTypeTok{centers =}\NormalTok{ i, }\DataTypeTok{nstart =} \DecValTok{30}\NormalTok{)}
\NormalTok{  table2[[i]]=}\KeywordTok{table}\NormalTok{(iris_data}\OperatorTok{$}\NormalTok{Species,model_all}\OperatorTok{$}\NormalTok{cluster)}
  \KeywordTok{print}\NormalTok{(table2[[i]])}
\NormalTok{\}}
\end{Highlighting}
\end{Shaded}

\begin{verbatim}
##             
##               1  2
##   setosa      0 50
##   versicolor 50  0
##   virginica  50  0
##             
##               1  2  3
##   setosa      0 50  0
##   versicolor 48  0  2
##   virginica   4  0 46
##             
##               1  2  3  4
##   setosa      0  0  0 50
##   versicolor 42  8  0  0
##   virginica   0 23 27  0
##             
##               1  2  3  4  5
##   setosa      0 50  0  0  0
##   versicolor 24  0  0 23  3
##   virginica   4  0 26  0 20
\end{verbatim}

\begin{Shaded}
\begin{Highlighting}[]
\CommentTok{#Tesing the algorithm with Petal Width & Sepal Width for clusters between (2-5)}
\NormalTok{table3<-}\KeywordTok{list}\NormalTok{()}
\ControlFlowTok{for}\NormalTok{ (i }\ControlFlowTok{in} \DecValTok{2}\OperatorTok{:}\DecValTok{5}\NormalTok{)\{}
\NormalTok{  model_all<-}\KeywordTok{kmeans}\NormalTok{(}\DataTypeTok{x=}\NormalTok{scale_iris[,}\KeywordTok{c}\NormalTok{(}\DecValTok{2}\NormalTok{,}\DecValTok{4}\NormalTok{)], }\DataTypeTok{centers =}\NormalTok{ i, }\DataTypeTok{nstart =} \DecValTok{30}\NormalTok{)}
\NormalTok{  table3[[i]]=}\KeywordTok{table}\NormalTok{(iris_data}\OperatorTok{$}\NormalTok{Species,model_all}\OperatorTok{$}\NormalTok{cluster)}
  \KeywordTok{print}\NormalTok{(table3[[i]])}
\NormalTok{\}}
\end{Highlighting}
\end{Shaded}

\begin{verbatim}
##             
##               1  2
##   setosa      1 49
##   versicolor 50  0
##   virginica  50  0
##             
##               1  2  3
##   setosa     49  0  1
##   versicolor  0 15 35
##   virginica   0 39 11
##             
##               1  2  3  4
##   setosa     16  1 33  0
##   versicolor  0 35  0 15
##   virginica   0 11  0 39
##             
##               1  2  3  4  5
##   setosa      0 16  1 33  0
##   versicolor 29  0 18  0  3
##   virginica  19  0  6  0 25
\end{verbatim}

\begin{Shaded}
\begin{Highlighting}[]
\CommentTok{#By comparing all the tables its clear that 3 clusters can define the data well, }
\CommentTok{#with Petal length and Petal width providing the best classification with }
\CommentTok{#Sertosa in cluster 3 completely without any misclassification. For Versicolor}
\CommentTok{#most of them are in cluster 2 with only 2 in cluster 1. Similarly,for virginica}
\CommentTok{#most of them are in cluster 1 with only 4 in cluster 2. The conclusions are }
\CommentTok{#made using table 2 using k=3 clusters.}
\end{Highlighting}
\end{Shaded}

\end{document}
